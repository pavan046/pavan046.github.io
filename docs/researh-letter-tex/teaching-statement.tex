%%%%%%%%%%%%%%%%%%%%%%%%%%%%%%%%%%%%%%%%%
% Plain Cover Letter
% LaTeX Template
% Version 1.0 (28/5/13)
%
% This template has been downloaded from:
% http://www.LaTeXTemplates.com
%
% Original author:
% Rensselaer Polytechnic Institute 
% http://www.rpi.edu/dept/arc/training/latex/resumes/
%
% License:
% CC BY-NC-SA 3.0 (http://creativecommons.org/licenses/by-nc-sa/3.0/)
%
%%%%%%%%%%%%%%%%%%%%%%%%%%%%%%%%%%%%%%%%%

%----------------------------------------------------------------------------------------
%	PACKAGES AND OTHER DOCUMENT CONFIGURATIONS
%----------------------------------------------------------------------------------------

\documentclass[12pt]{letter} % Default font size of the document, change to 10pt to fit more text

\usepackage{newcent} % Default font is the New Century Schoolbook PostScript font 
%\usepackage{helvet} % Uncomment this (while commenting the above line) to use the Helvetica font

% Margins
\topmargin=-0.7in % Moves the top of the document 1 inch above the default
\textheight=9in % Total height of the text on the page before text goes on to the next page, this can be increased in a longer letter
\oddsidemargin=-10pt % Position of the left margin, can be negative or positive if you want more or less room
\textwidth=7in % Total width of the text, increase this if the left margin was decreased and vice-versa

%\let\raggedleft\raggedright % Pushes the date (at the top) to the left, comment this line to have the date on the right

\begin{document}

%----------------------------------------------------------------------------------------
%	ADDRESSEE SECTION
%----------------------------------------------------------------------------------------

%\begin{letter}{} 

%----------------------------------------------------------------------------------------
%	YOUR NAME & ADDRESS SECTION
%----------------------------------------------------------------------------------------
{\large\bf Teaching Statement} \\ % Your name
Pavan Kapanipathi, Kno.e.sis Center, Wright State University, USA (pavan@knoesis.org)

\noindent\makebox[\linewidth]{\rule{\linewidth}{0.4pt}}
\vspace{0.1in}

I had the privilege of being a teaching assistant for "Introduction to Java" during my graduate studies and also mentored 6 students in my lab. This experience is one of the primary motivations for me to pursue an academic career. I find it immensely satisfying that as a professor I would have the opportunity to interact and educate as well as learn from bright students with enquiring minds who will shape the future of technology.

Students in field such as computer science with significant real-world applications and impact should have hands-on experience in classes. \textit{``Learning-by-doing"} is a skill that students should acquire starting from classrooms. The courses I would teach will have an emphasis on projects that are designed to address problems relevant to the real-world. While this would prepare the students for industry jobs, it would also reduce the STEM skill gap in the industry as evident from a recent survey involving major companies []. With this in mind, I would encourage  interactive and participatory learning. I would follow a ‘flipped classroom approach’ where the students come prepared on assigned topics each class by reading the textbook and/or other supplementary learning tools, such as audio-visual data. This allows the classroom time to be  utilized for interactive learning by just-in-time lecturing (lectures to address student doubts/concerns), peer learning, discussions, and hands-on learning sessions. Research has shown that this kind of participatory learning has a significant impact on students understanding of concepts and in turn their success []. 


In today's world, computer science is making a significant impact in conjunction with other disciplines such as healthcare, disaster management, and economics, and I intend to consciously discuss and demonstrate the value of interdisciplinary engagement with the students- both in the classroom and through their involvement in class and research projects..
I intend to guide the students  to acquire knowledge of other domains without restricting themselves to only the fundamentals of computer science. Learning these interdisciplinary skills also involves having students communicate, and collaborate in a team with people from different disciplines. To enable and teach such skills, my courses would include projects that are proposed, designed, and implemented by students on various real-world datasets. These effort will also be closely aligned with the broader impact and educational impact components of my NSF proposals.  Students will have to understand the datasets of different disciplines forcing them to acquire necessary knowledge of another domain. Furthermore, the proposal and report of the project will provide opportunity for students to learn to communicate their ideas and projects clearly, particularly in writing.     

\vspace{1.0em} 
\textbf{Curriculum Development.}
\vspace{0.3em}

Computing has become a necessity in a variety of disciplines to solve significant challenges and discover new knowledge. The job market already put  increasing  emphasis on graduates with applied skills to perform tasks[3]. I intend to develop courses that are theoretically grounded but which mostly focus on hands-on learning. I want to start the following curriculum to further enrich the existing \textbf{XXXX} program in the university for both the undergraduate (UG) and graduate (PG) levels and to better prepare students for upcoming industry demands. For the betterment of the students, I will also make adjustments from semester to semester based on the student feedback. In the course, I plan to strongly encourage team and individual projects where the students would prepare a project plan and implement a prototype. The teams will be evaluated via class demonstrations and reports.

\begin{itemize}
\item \textit{Recommender Systems (UG and PG)}: Netflix announced a challenge with prize money of \$1Million to develop an algorithm that can outperform their movie recommendation system. Most major internet companies bank on recommendation systems for significant revenue generation. In this era, it is necessary for students to understand and learn technologies that are playing an immense role on the web. This course will cover some fundamentals of linear algebra, and machine learning and the recommendation systems will span across entities (movies, music), semi-structured (ads), and unstructured text (tweets, and news articles). Projects would play a significant role in evaluation and the focus of projects will be towards research problems as well as the development of smart recommendation systems.

   
\item \textit{Knowledge-Graphs and Semantic Web (UG and PG):} Knowledge graphs are extensively used by the industry (e.g., Google, Microsoft, IBM, Yahoo!) and will continue to grow. This course would emphasize some fundamentals of web technologies, graph theory, and the semantic web. The course would also include: (1) building/creating knowledge graphs from unstructured and semi-structured data sources such as (news articles and Wikipedia); (2) use of the created knowledge graphs and open graph datasets (Wikidata, Linked Open Data) for multiple applications (some domain-specific). The objective is to prepare students with skills in state-of-the-art Web technologies

\item \textit{Social Data and Network Analysis (UG and PG): } Social networking platforms have gained popularity in recent years. The data generated on these platforms are extensively used by a spectrum of domains to tap into the wisdom of the crowd. The domains ranges from healthcare and journalism to the economy and finance. Textual data and network data on these platforms complement each other to perform analysis and gain insights. In this course, I intend to cover basics of both text mining and network analysis on social network data. This would also introduce the students to algorithms and state-of-the-art techniques that input both modalities of data (network and text) for analysis.  
\end{itemize}

\vspace{1.0em} 
\textbf{Teaching Courses.}
\vspace{0.3em}

In addition to developing new courses and programs that reflect current and emerging technological changes in the field, I look forward the opportunity to teach existing courses in the department including but not limited to the following:

\begin{itemize}
\item Information Retrieval
\item Data Mining
\item Web Programming
\item \textbf{FILL MORE}
\end{itemize}

I believe that my teaching philosophy, approach, and experience in the mentioned areas will allow me to contribute significantly towards meeting the education goals of the \textbf{XXXXX}.

\end{document}

\end{document}