% resume.tex
%

\documentclass[letterpaper,11pt]{article}

%-----------------------------------------------------------
\usepackage[empty]{fullpage}
\usepackage{color}
\usepackage{multicol}
\usepackage{hyperref}
\definecolor{mygrey}{gray}{0.80}
\raggedbottom
\raggedright
\setlength{\tabcolsep}{0in}

% Adjust margins to 0.5in on all sides
\addtolength{\oddsidemargin}{-0.5in}
\addtolength{\evensidemargin}{-0.5in}
\addtolength{\textwidth}{1.0in}
\addtolength{\topmargin}{-0.5in}
\addtolength{\textheight}{1.0in}

%-----------------------------------------------------------
%Custom commands
\newcommand{\resitem}[1]{\item #1 \vspace{-2pt}}
\newcommand{\resheading}[1]{{\large \colorbox{mygrey}{\begin{minipage}{\textwidth}{\textbf{#1 \vphantom{p\^{E}}}}\end{minipage}}}}
\newcommand{\ressubheading}[4]{
\begin{tabular*}{7.0in}{l@{\extracolsep{\fill}}r}
		\textbf{#1} & #2 \\
		\textit{#3} & \textit{#4} \\
\end{tabular*}\vspace{-6pt}}
%-----------------------------------------------------------


\begin{document}

\begin{tabular*}{7.5in}{l@{\extracolsep{\fill}}r}
\textbf{\large Pavan Kapanipathi}  & +1 713 435 9982 (phone)\\
 Knoe.s.is Center, Wright State University &  pavan[at]knoesis.org \\
 3640 Colonel Glenn Highway, Dayton, OH 45435 &  \url{http://knoesis.org/researchers/pavan}\\
\end{tabular*}
\\

\vspace{0.1in}
\resheading{Research Interests}
\begin{itemize}
\item Information Extraction, Contextual Filtering, Semantic Web, Personalization, User Modeling, Recommendation
\end{itemize}
\resheading{Education}
\begin{itemize}
\item
	\ressubheading{Wright State University}{Dayton, OH}{Phd, Computer Science}{2012 - Present}

\item
	\ressubheading{Wright State University}{Dayton, OH}{Masters, Computer Science}{2009 - 2012}

\item
	\ressubheading{Visvesvaraya Technological University}{Karnataka, India}{Bachelor of Engineering in Computer Science}{2003 - 2007}

\end{itemize}

\resheading{Experience}
\begin{itemize}
\item
	\ressubheading{Dept of Computer Science and Engg, Wright State University}{Dayton, OH, USA}{Graduate Teaching/Research Assistant}{January 10 - Present}
	\linebreak
\item
\ressubheading{Samsung Research America}{San Jose, US}{Research Intern}{May 13 - December 31}
	\linebreak
\item 
\ressubheading{IBM TJ Watson Research Center}{New York, US}{Research Intern}{May 13 - August 13}
	\linebreak
\item
	\ressubheading{Digital Enterprise Research Institute}{Galway, Ireland}{Research Intern}{April 11 - August 11}
	\linebreak
\item
	\ressubheading{Accenture}{Bangalore, India}{Senior Programmer}{July 2007 - March 2009}
	\linebreak
     
\item
	\ressubheading{Robert Bosch}{Bangalore, India}{Project Trainee}{Feb 2007 - June 2007}
	\linebreak
     
\end{itemize}
\resheading{Publications and Patents}
\begin{itemize}
\item \textbf{Work in Progress}
\begin{itemize}
\item[-] \textbf{Pavan Kapanipathi}, Krishnaprasad Thirunarayan, Fabrizio Orlandi, Amit Sheth, Pascal Hitzler. A Real-Time \#approach for Continuous Crawling of Events on Twitter by Leveraging Wikipedia. Technical Report (In Progress).
\item[-] \textbf{Pavan Kapanipathi}, Prateek Jain, Chitra Venkataramani, Amit Sheth, Derek Doran. Hierarchical Knowledge Bases to Identify User Interests on Social Media. (Journal in Progress).
\item[-] Revathy Krishnamurthy, \textbf{Pavan Kapanipathi}, Amit Sheth, Krishnaprasad Thirunarayan. Location Prediction of Twitter users using Wikipedia (Submitted for publication).
\item[-] Edwin Heredia, Joakhim Soderberg, \textbf{Pavan Kapanipathi}, Glenn Algie, Rodrigo Laiola Guimaraes, Alan Messer. Semantically Enriched Story Compositions from Geo-Temporal Logs. (Submitted for publication)
\end{itemize}
\item\textbf{Published Work}
\begin{itemize}
\item[-] \textbf{Pavan Kapanipathi}, Prateek Jain, Chitra Venkataramani, Amit Sheth. User Interests Identification on Twitter Using a Hierarchical Knowledge Base. Extended Semantic Web Conference 2014, Crete Greece \textbf{(23\% acceptance)}
\item[-] \textbf{Pavan Kapanipathi}, Prateek Jain, Chitra Venkataramani, Amit Sheth. Hierarchical Interest Graph from Twitter.  23rd International conference on World Wide Web companion 2014 (WWW 2014), Seoul, South Korea. 
\item[-] Fabrizio Orlandi, \textbf{Pavan Kapanipathi}, Alexandre Passant, Amit Sheth. Characterising concepts of interest leveraging Linked Data and the Social Web. The 2013 IEEE/WIC/ACM International Conference on Web Intelligence, Atlanta, USA, United States, 2013.
\item[-] \textbf{Pavan Kapanipathi}, Fabrizio Orlandi, Amit Sheth, Alexandre Passant. Personalized Filtering of the Twitter Stream. \textit{2nd workshop on Semantic Personalized Information Management at ISWC 2011}, September 2011. 
\item[-] \textbf{Pavan Kapanipathi}, Julia Anaya, Amit Sheth, Brett Slatkin, Alexandre Passant. Privacy-Aware and Scalable Content Dissemination in Distributed Social Network. \textit{10th International Semantic Web Conference 2011}, Bonn, Germany,  September 2011. \textbf{(acceptance rate 22\%)}
\item[-] \textbf{Pavan Kapanipathi}, Julia Anaya, Alexandre Passant . SemPuSH: Privacy-Aware and Scalable Broadcasting for Semantic Microblogging. 
\textit{10th International Semantic Web Conference 2011}, Bonn, Germany, September 2011.
\item[-]  Alexandre Passant, Julia Anaya, Owen Sacco, \textbf{Pavan Kapanipathi}. SMOB: The Best of Both Worlds. 
\textit{Federated Social Web Europe Conference}, Berlin, June 3rd -5th 2011.
\item[-] Alexandre Passant, Owen Sacco, Julia Anaya, \textbf{Pavan Kapanipathi}. Privacy-By-Design in Federated Social Web Applications. 
\textit{Websci 2011}, Koblenz, Germany June 14-17, 2011.
	\item[-] Pablo Mendes, \textbf{Pavan Kapanipathi}, Alexandre Passant. Twarql: Tapping into the Wisdom of the Crowd. \textit{Triplification Challenge 2010 at 6th International Conference on Semantic Systems (I-SEMANTICS)}, Graz, Austria, 1-3 September 2010. \textbf{(Winner of Triplification Challenge 2010)}
  \item[-] Pablo Mendes, Alexandre Passant, \textbf{Pavan Kapanipathi}, Amit Sheth. Linked Open Social Signals.\textit{WI2010 IEEE/WIC/ACM International Conference on Web Intelligence (WI-10)}, Toronto, Canada, Aug. 31 to Sep. 3, 2010. \textbf{(16.6\% acceptance)}
  \item[-] Pablo Mendes, \textbf{Pavan Kapanipathi}, Delroy Cameron, Amit Sheth. Dynamic Associative Relationships on the Linked Open Data Web. \textit{In: Proceedings of the WebSci10: Extending the Frontiers of Society On-Line}, April 26-27th, 2010, Raleigh, NC: US.      
\end{itemize}
\item \textbf{Patents}
\begin{itemize}
\item Edwin Heredia, Joakhim Soderberg, \textbf{Pavan Kapanipathi}, Glenn Algie, Rodrigo Laiola Guimaraes, Alan Messer. Semantically Enriched Story Compositions from Geo-Temporal logs
\end{itemize}
\end{itemize}
\resheading{Research Projects}
\begin{itemize}
\item 
\ressubheading{User Interest Graph from Social Data}{Collaboration - IBM Research, DERI}{\url{http://bit.ly/hierarchical-interest-graph}}{}
	\linebreak
	\begin{itemize}
\item \textit{\textbf{Summary}} Industry and researchers have identified numerous ways to monetize microblogs for personalization and recommendation. A common challenge across these different works is identification of user interests. Although techniques have been developed to address this challenge, we focus on approaches that leverages entities and their relationships to extend the state of art personalization techniques. In our research, we use twitter data with semantic web technologies and graph techniques to determine, expand user interest entities, in-turn generating a user interest graph. 
\end{itemize}

\item
	\ressubheading{Continuous Semantic Crawling of Events on Twitter}{Collaboration - DERI, Ireland}{\url{http://bit.ly/continuous-crawling}}{}
	\linebreak
	\begin{itemize}
\item \textit{\textbf{Summary}} The need to tap into the wisdom of the crowd via social networks in real-time has already been demonstrated during critical events such as the Arab Spring and the recently concluded US Elections. As Twitter becomes a platform of choice for streaming event related information in real-time, we face several challenges in the related to filtering, realtime monitoring and tracking of the dynamic evolution of an event. We present a novel approach to continuously track an evolving event on Twitter by leveraging hashtags that are filtered using an evolving background knowledge (Wikipedia).
\end{itemize}

\item
\ressubheading{Twarql: Twitter feeds through SPARQL}{Collaboration - DERI, Ireland}{\url{http://wiki.knoesis.org/index.php/Twarql}}{}
\linebreak
\begin{itemize}
    \item \textit{\textbf{Summary}} Twitter has become a prominent medium to share opinions, observations and suggestions in real-time. Insights from these microposts ("Wisdom of the Crowd") has proved to be invaluable for businesses and researchers around the world. However, the microblog data published is increasing in numbers with the popularity and growth of Twitter. This has induced challenges in filtering these microblog data to cater the needs for aggregation and collective analysis for sensemaking. Twarql addresses these challenges by leveraging Semantic Web technologies to enable a flexible query language for filtering microblog posts.
\end{itemize}

\item
\ressubheading{SemPUSH: Controlled Content Dissemination in Social Networks}{Collaboration - Google}{}{DERI Internship}
\begin{itemize}
\item[-] \textit{\textbf{Summary}} Users of traditional microblogging platforms such as Twitter face drawbacks in terms of (1) Privacy of status updates as a followee -- reaching undesired people (2) Information overload as a follower -- receiving uninteresting microposts from followees. In this project we have implemented a privacy-aware version of google's PuSH protocol (Semantic Hub) for distributed and user-controlled dissemination of microposts using SMOB (semantic microblogging framework). The approach leverages users' Social Graph to dynamically create group of followers who are eligible to receive micropost. The restrictions to create the groups are provided by the followee based on the hastags in the micropost. Both SMOB and Semantic Hub are available as open source
\end{itemize}


\item
\ressubheading{Twitris+: 360 degree Social Media Analytics platform}{}{\url{http://twitris.knoesis.org/}}{}
\linebreak
\begin{itemize}
    \item \textit{\textbf{Summary}} Users are sharing voluminous social data (800M+ active Facebook users, 1B+ tweets/week) through social networking platforms accessible by Web and increasingly via mobile devices. This gives unprecedented opportunity to decision makers-- from corporate analysts to coordinators during emergencies, to answer questions or take actions related to a broad variety of activities and situations: who should they really engage with, how to prioritize posts for actions in the voluminous data stream, what are the needs and who are the resource providers in emergency event, how is corporate brand performing, and does the customer support adequately serve the needs while managing corporate reputation etc. We demonstrate these capabilities using Twitris+ by multi-faceted anlaysis along dimensions of Spatio-Temporal-Thematic (STT), People-Content-Network (PCN), and Subjectivity: Emotion-Sentiment-Intent (ESI). Twitris' diversity and depth of analysis is unprecedented. Twitris v1 [2009] focused on STT, Twitris v2 [2011] focused on PCN, and Twitris v3 [2012- ] initiated ESI, extended other dimensions by extending PAN analysis with expression capability involving use of background knowledge, and will soon add real-time analytics incorporating Kno.e.sis' Twarql framework.

Twitris leverages an array of techniques and technologies that traditionally fall under big data (or scalable unstructured data analysis), social media analysis (including user generated content analysis), and Semantic Web (including extensive use of RDF), and algorithms that use statistical, linguistics, machine learning, and complex/semantic query processing. 
\end{itemize}
\end{itemize}

\resheading{Research Grants and Proposals}
\begin{itemize}
\item NIH-R01 - (3 years) \textit{Significant Contribution}
\begin{itemize}
\item[-] Title: Trending: Social media analysis to monitor cannabis and synthetic cannabinoid use
\item[-] Awardees: Wright State University (Kno.e.sis, CITAR), University of Massachusetts - Amherst 
\item[-] Project: EdrugTrends
\end{itemize}
\end{itemize}

\resheading{Skills}
\linebreak
\linebreak
	\textbf{Programming:} Java, C++, SQL, Python, R.\linebreak 
	\textbf{Familiar:} C\#, VC++, Perl,  Hadoop (MapReduce), Pig(Hadoop).\linebreak
    \textbf{Web:} PHP, HTML, Javascript, Servlets, CSS, RDF, SPARQL. \linebreak
    \textbf{Tools \& Softwares:} MySQL, Apache, Tomcat, Virtuoso, Jena, svn, git. \linebreak
\linebreak

\resheading{Professional Activities and Achievements}
\begin{itemize}
\item Invited to present my research for Wright State University, Graduate School External Advisory Board.
\item Invited talk Frontiers of Cloud Computing and Big Data Workshop, 2014, organized by IBM TJ Watson Research Center. 
\item Awarded scholarship of \$1500 to attend the 10th International Semantic Web Conference at Bonn, Germany.
\item Winner of the Triplification Challenge: ``Twarql: Tapping into the Wisdom of the Crowd" won the Triplification Challenge in the open track at 6th International Conference on Semantic Systems in September 2010. 
\item External Reviewer: ICWSM2015, WWW2015, AAAI2013, ESWC2013, WWW2013, ISWC2011, IJSWIS.
\item PC Member: HT2015, ESWC2015, ESWC2014, WI2014, WoLE2013(WWW Workshop), WI2013, DERIVE2012 (ISWC Workshop).
\end{itemize}

\resheading{References}

\begin{multicols}{2}
\textbf{\large Amit Sheth}\\
Director, Knoe.s.is Center, Wright State University\\
Dayton, OH \\
email:amit[at]knoesis.org\\
\columnbreak
\textbf{\large Krishnaprasad Thirunarayan}\\
Professor, Wright State University\\
Dayton, OH \\
email:tkprasad[at]knoesis.org\\
\end{multicols}

\begin{multicols}{2}
\textbf{\large Prateek Jain}\\
Data Scientist, IgnitionOne Solutions\\
New York, US\\
email:jainpr[at]us.ibm.com\\
\columnbreak
\textbf{\large Pascal Hitzler}\\
Associate Professor, Wright State University\\
Dayton, OH \\ 
email:pascal[at]knoesis.org\\
\end{multicols}


%\end{itemize}
%\end{itemize}

\end{document}
